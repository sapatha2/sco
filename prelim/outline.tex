\documentclass{article}
\title{Prelim Outline}
\author{Shivesh Pathak}
\usepackage[margin=1.0in]{geometry}
\begin{document}
\maketitle

\section{Introduction and Methods}
\textit{The goal of the introduction is to show the reader that the project is valuable. Should be x pages, other 5-x page will be methods.}
\paragraph{Overall goal}
We want to understand the low energy behavior of the cuprate superconductors.
\\
\textit{This paragraph will be outlining the various phenomenon attributed with the cuprates, e.g. the many phases.}

\paragraph{Barrier to achieving goal}
The strong coupling between magnetic, charge, and lattice degrees of freedom in the cuprates makes writing down an accurate low energy effective theory very challenging.
\\
\textit{This paragraph will outline the many experiments that show the strong coupling between these many degrees of freedom.}

\paragraph{Current state of the art}
Many theoretical approaches to generating a low energy model for the cuprates have been attempted with varying degrees of success.
\\
\textit{This paragraph will outline the various approaches people have tried to generate effective theories. Hole doped, electron doped, pen and paper, computational, energetics, one- and two-body properties.}

\paragraph{Advancement to the state of the art}
We propose that a density matrix downfolding approach using \textit{ab initio} quantum Monte Carlo calculations can generate an effective model for the infinite layer electron-doped cuprates at zero temperature which can accurately describe low-energy properties and energetics. 
\\
\textit{This paragraph will describe in detail each of the conditions in the thesis sentence.}

\paragraph{Density matrix down folding} 
The density matrix downfolding method allows us to downfold a first-principles Hamiltonian into a low-energy effective theory by mapping the downfolding problem to a linear regression problem which uses \textit{ab-initio} energies and reduced density matrices of wave functions sampled from a low-energy subspace of the full Hilbert space.
\\
\textit{This paragraph should outline the details of the DMD method. No reference to QMC yet.}

\paragraph{Fixed node diffusion Monte Carlo}
We will use fixed node diffusion Monte Carlo (FN-DMC) to generate wave functions in our chosen low-energy subspace and to accurately calculate \textit{ab-initio} energies and reduced density matrices.
\\
\textit{This paragraph should outline how we will use FN-DMC (with DFT) to generate the low energy states, their energies, and their RDMs}


\end{document}